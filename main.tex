\documentclass[oneside,12pt]{book}

% Package dependencies
\usepackage{geometry}
\usepackage{lipsum}
\geometry{left=32mm, right=30mm, bottom=25mm, top=25mm}
\usepackage{amsmath , amsthm , amssymb}
\usepackage{float}
\usepackage{graphicx}
\usepackage{hyperref}
\usepackage{apacite}
\usepackage{scrextend}
\usepackage{blindtext}
\usepackage{fancyhdr}
\usepackage{multicol}
\usepackage{csquotes}
\usepackage{bm}
\usepackage{subcaption}
\usepackage{tocbibind}
\usepackage{pdfpages}
\usepackage[toc,page]{appendix}
\usepackage{wrapfig}
\usepackage{colortbl}
\usepackage{multirow}

% document formatting
\addtokomafont{labelinglabel}{\sffamily}
\setlength{\columnsep}{1cm}
\renewcommand{\baselinestretch}{1.5}


\begin{document}

% Title page
\begin{titlepage}
    \begin{center}
        \vspace*{1.5cm}
        
        \Huge
        \textbf{Thesis Title}
        
        \vspace{0.5cm}
        \begin{figure}[H]
    	\centering
    	\hspace{7mm} \includegraphics[scale=0.5]{TU_logo}
        \end{figure}
        
        \vspace{1.5cm}
        
        \textbf{Student Name}
        
        \vfill
       \large
        A dissertation submitted in partial fulfilment of the requirements of\\
	Dublin Institute of Technology for the degree of\\
	M.Sc. in Computing (Stream)\\
       \vspace{0.5cm}
        \textbf{Date}
        \vspace{0.8cm}
 
    \end{center}
\end{titlepage}

% Define headers / footer style for the whole doc
\pagestyle{fancy}
\fancyhf{}
\fancyfoot[CE,CO]{\thepage}
\renewcommand{\headrulewidth}{0pt}

% APA style for referencing
\bibliographystyle{apacite}

% Roman numerals for the 'administrative' sections
\pagenumbering{Roman}

% Declaration page
\chapter*{Declaration}
\addcontentsline{toc}{chapter}{Declaration}
I certify that this dissertation which I now submit for examination for the award of
MSc in Computing (Stream), is entirely my own work and has not been taken
from the work of others save and to the extent that such work has been cited and
acknowledged within the text of my work.
\\
\\
This dissertation was prepared according to the regulations for postgraduate study of
the Technological University Dublin and has not been submitted in whole or part for an
award in any other Institute or University.
\\
\\
The work reported on in this dissertation conforms to the principles and requirements
of the Institute’s guidelines for ethics in research.
\vfill
\noindent
\textit{\textbf{Signed:}}  \\

\noindent
\textit{\textbf{Date:}}
\vspace{0.8cm}

% Abstract
\chapter*{Abstract}
\addcontentsline{toc}{chapter}{Abstract}
\par Abstract outlining the project
\\
\vfill
\noindent
\textbf{Keywords:} \quad Keyword 1, keyword 2

%Acknowledgements
\chapter*{Acknowledgments}
\addcontentsline{toc}{chapter}{Acknowledgments}

% Contents and lists
\newpage
\tableofcontents

\listoffigures

\listoftables

\chapter*{List of Acronyms}
\addcontentsline{toc}{chapter}{List of Acronyms}
\begin{table}[H]
  \centering
    \begin{tabular}{ l l }
    \textbf{ACRONYM1} & Meaning \\
    \textbf{ACRONYM2} & Meaning \\ 
    \end{tabular}
\end{table}

\newpage

% Standard numbering starts from here
\pagenumbering{arabic}
\fancyhead[RE,LO]{\leftmark}
\renewcommand{\headrulewidth}{2pt}


% Start of core thesis content

%Introduction chapter
% ============================================================= %
\chapter{Introduction}
\section{Background}
\par This content gives some examples of latex commands that might be used. To cite an article, using the apa style, at the end of a sentence, simply reference the label from the bibliography file. Examples below:
\begin{itemize} 
\item phd thesis: \cite{Longo2014}.
\item article in a conference proceeding: \cite{Longo2016}
\item journal article: \cite{Longo2015a}
\item chapter book: \cite{Longo2011c}

if you wanna cite something at the beginning of the sentence see below.\\

\citeA{Longo2016} proposed a framework for human mental workload representation and assessment.


\end{itemize}
\par An example of a numbered bullet point list. To number the list, replace itemize with enumerate
\begin{enumerate}
\item This is the first item.
\item Item number 2.
\item etc.
\end{enumerate}

\par An example of a simple table. This can be referenced in the text using the label given to table \ref{tab:sample_table}. The same can be done for sections, equations or figures.
\begin{table}[H]
  \centering
    \begin{tabular}{| c | c | c |}
    \hline
     & \textbf{Column 1} & \textbf{Column 2} \\ \hline
    \textbf{Row 1} & 0 & 1 \\ \hline
    \textbf{Row 2} & 2 & 3 \\ \hline
    \textbf{Row 3} & 4 & 5 \\ \hline
    \end{tabular}
  \caption{A sample table}
  \label{tab:sample_table}
\end{table}

\par An example of a figure or image:

\begin{figure}[H]
    \centering
    \includegraphics[scale=0.5]{TU_logo}
    \caption{Sample image}
    \label{fig:logo}
\end{figure}

\par A sample equation:

\begin{equation}
e^{i\pi} = -1
\end{equation}

\section{Research Project/problem}

\section{Research Objectives}

\section{Research Methodologies}

\section{Scope and Limitations}

\section{Document Outline}


% Literature review chapter
% ============================================================= %
\chapter{Review of existing literature}

\section{A section}

\subsection{A sub-section}

\subsubsection{A sub-sub-section}

\par Some content.



% Design and methodology chapter
% ============================================================= %
\chapter{Experiment design and methodology}



% Results, evaluation and discussion chapter
% ============================================================= %
\chapter{Results, evaluation and discussion}


% Concluding chapter
% ============================================================= %
\chapter{Conclusion}
\section{Research Overview}

\section{Problem Definition}

\section{Design/Experimentation, Evaluation \& Results}

\section{Contributions and impact}

\section{Future Work \& recommendations}


% End of thesis content
% ============================================================= %

% Include the bibligraphy by referencing the correct .bib file
\bibliography{sample_bibliography}

% Optional appendices
\appendix
\chapter{Additional content}

\end{document}